%%%%%%%%%%%%%%%%%%%%%%%%%%%%%%%%%%%%%%%%%%%%%%%%%%%%%%%%%%%%%%%%%%%%%%%%%%%%%%%%
% ZÁKLADNÍ NASTAVENÍ DOKUMENTU
%%%%%%%%%%%%%%%%%%%%%%%%%%%%%%%%%%%%%%%%%%%%%%%%%%%%%%%%%%%%%%%%%%%%%%%%%%%%%%%%
\documentclass[a4paper, 12pt]{article}

% Balíčky pro kódování a češtinu
\usepackage[utf8]{inputenc} % Správné kódování (UTF-8)
\usepackage[T1]{fontenc}    % Moderní kódování fontů
\usepackage[czech]{babel}   % České dělení slov a názvy (Obsah, Kapitola...)

% Geometrie stránky (okraje)
\usepackage[a4paper, margin=2.5cm]{geometry}

% Matematické symboly (pro L*a*b* a šipku)
\usepackage{amsmath}
\usepackage{amssymb} % Pro \to

% Balíček pro odstavce bez odsazení a s mezerou mezi nimi
\usepackage{parskip}

% Balíček pro formátování kódu (MATLAB a Python)
\usepackage{listings}
\usepackage{xcolor} % Pro barvy (např. rámečku kódu)

\usepackage{graphicx}   % Pro vkládání obrázků
\usepackage{subcaption} % Pro obrázky vedle sebe s podpopisky (a), (b)

% Definice šedé barvy pro rámeček
\definecolor{framegray}{rgb}{0.8,0.8,0.8}

% Nastavení stylu pro 'listings'
\lstset{
    basicstyle=\ttfamily,        % Použití neproporcionálního písma
    frame=tb,                    % Rámeček pouze nahoře (top) a dole (bottom)
    rulesepcolor=\color{framegray}, % Barva čáry rámečku
    framerule=0.5pt,             % Tloušťka čáry
    numbers=none,                % Bez číslování řádků
    breaklines=true,             % Automatické lámání dlouhých řádků
    keepspaces=true,             % Zachovat mezery
    showstringspaces=false,      % Nezobrazovat speciální znaky pro mezery v řetězcích
    keywordstyle=\color{blue},   % Barva klíčových slov (pokud by byla definována)
    stringstyle=\color{purple},  % Barva řetězců (pokud by byla definována)
    commentstyle=\color{gray},    % Barva komentářů (pokud by byla definována)
}

% Pro správné vkládání názvů souborů s podtržítky
% \texttt{} se postará o správné písmo a zobrazení podtržítka
\newcommand{\filename}[1]{\texttt{#1}}

%%%%%%%%%%%%%%%%%%%%%%%%%%%%%%%%%%%%%%%%%%%%%%%%%%%%%%%%%%%%%%%%%%%%%%%%%%%%%%%%
% ZAČÁTEK DOKUMENTU
%%%%%%%%%%%%%%%%%%%%%%%%%%%%%%%%%%%%%%%%%%%%%%%%%%%%%%%%%%%%%%%%%%%%%%%%%%%%%%%%
\begin{document}

\section*{Část B}

Část B úlohy byla řešena v rámci dvou skriptů: \filename{segmentace\_std.m} řeší základní úlohu a bonusovou úlohu s využitím STD filtru, pak \filename{GeoPackageUTM.py} řeší druhou bonusovou úlohu související s exportem výsledné masky do GeoPackage.

\subsection*{Popis skriptu \filename{segmentace\_std.m}}

Tento skript provádí kompletní proces segmentace lesních ploch z topografické mapy, od přípravy dat až po finální čištění masky a její export. Pro analýzu textury byla nejprve zvažována banka Gaborových filtrů (4 směry, 3 frekvence). Tento přístup by však spolu s barevnými kanály vytvořil 14 příznaků pro \texttt{imsegkmeans()}, což se ukázalo jako výpočetně náročné a vedlo k nežádoucímu přesegmentování. Proto byla pro finální řešení zvolena efektivnější filtrace pomocí standardní odchylky \texttt{stdfilt()}, která texturu lesa popsala jediným robustním příznakem.

% --- Krok 1 ---
\subsubsection*{Krok 1: Načtení a převod barev}

\paragraph{Kód:}
\noindent
\noindent
\begin{lstlisting}[language=Matlab]
labImage = rgb2lab(imread("TM25_sk1.jpg"));
\end{lstlisting}

\paragraph{Popis:}
Nejprve se načte zdrojový obrázek mapy (\filename{TM25\_sk1.jpg}). Okamžitě se převádí z barevného prostoru RGB do $L^*a^*b^*$.

\paragraph{Důvod:}
Prostor $L^*a^*b^*$ je pro segmentaci mnohem vhodnější, protože odděluje světlost ($L^*$) od barevných složek ($a^*$, $b^*$). To nám umožňuje analyzovat texturu nezávisle na barvě a barvu nezávisle na světlosti.

% --- Krok 2.1 ---
\subsubsection*{Krok 2.1: Příprava STD filtru}

\paragraph{Kód:}
\noindent
\begin{lstlisting}[language=Matlab]
L_channel = labImage(:,:,1);
a_channel = labImage(:,:,2);
stdFeatures = stdfilt(L_channel, nhood);
\end{lstlisting}

\paragraph{Popis:}
Tento krok připravuje „příznaky“ (features) pro segmentaci.
Ze $L^*a^*b^*$ obrázku se izoluje kanál $L^*$ (světlost) a kanál $a^*$ (zelená-červená). Kanál $a^*$ se uloží pro pozdější identifikaci lesa (který je zelený).
Na kanál $L^*$ se aplikuje filtr standardní odchylky (\texttt{stdfilt}) s okolím 9x9 pixelů (\texttt{nhood}).

\paragraph{Důvod:}
STD filtr je klíčový pro splnění bonusové části. Vytvoří nový obrázek, kde vysoké hodnoty znamenají vysokou lokální variabilitu (hrubou texturu, např. tečkovaný les) a nízké hodnoty znamenají hladkou plochu (pole, louky).

% --- Krok 2.2 ---
\subsubsection*{Krok 2.2: Příprava dat pro \texttt{imsegkmeans()}}

\paragraph{Kód:}
\noindent
\begin{lstlisting}[language=Matlab]
allFeatures = cat(3, abChannels, gaborMag);
features_scaled = zscore(features_2d);
\end{lstlisting}

\paragraph{Popis:}
Zde se připraví finální data pro segmentační algoritmus.
Vytvoří se „balík“ příznaků spojením dvou barevných kanálů ($a^*$, $b^*$) a jednoho texturového kanálu \texttt{stdFeatures}. Výsledkem je 3-kanálový obrázek, kde má každý pixel tři vlastnosti: barvu ($a^*$, $b^*$) a texturu (std).
Protože tyto tři kanály mají různé rozsahy hodnot, je nutné je normalizovat. Funkce \texttt{zscore} standardizuje každý příznak tak, aby měl průměr 0 a směrodatnou odchylku 1. Tím je zajištěno, že algoritmus k-Means dává všem příznakům stejnou váhu.
Je nastaven počet clusterů $k = 2$ (Les a Ne-les).

% --- Krok 3 ---
\subsubsection*{Krok 3: Segmentace}

\paragraph{Kód:}
\noindent
\begin{lstlisting}[language=Matlab]
pixelLabels = imsegkmeans(features_scaled_3d, k, ...);
\end{lstlisting}

\paragraph{Popis:}
Na normalizovaných, třípásmových datech (barva + textura) se spustí algoritmus k-Means.

\paragraph{Výsledek:}
Vznikne obrázek \texttt{pixelLabels}, kde má každý pixel přiřazené číslo (1 nebo 2) podle toho, do kterého clusteru byl zařazen.

% --- Krok 4 ---
\subsubsection*{Krok 4: Identifikace clusteru a maska}

\paragraph{Kód:}
\noindent
\begin{lstlisting}[language=Matlab]
for i=1:k
    clusterMeans(i) = mean(a_channel(pixelLabels==i));
end
[~, idLesa] = min(clusterMeans);
forestMask = (pixelLabels==idLesa);
\end{lstlisting}

\paragraph{Popis:}
Tento krok zjišťuje, který cluster (1 nebo 2) představuje les.
Vypočítá se průměrná hodnota $a^*$ kanálu pro oba clustery.
Cluster s nejnižší (nejvíce zápornou) průměrnou hodnotou $a^*$ je identifikován jako les (protože $a^*$ kanál kóduje zelenou barvu).
Vytvoří se binární maska (\texttt{forestMask}), která má hodnotu 1 (true) pro všechny pixely patřící do clusteru lesa a 0 (false) pro všechny ostatní.

% --- Krok 5.1 ---
\subsubsection*{Krok 5.1: Čištění šumu a tenkých čar}

\paragraph{Kód:}
\noindent
\begin{lstlisting}[language=Matlab]
cleanMask = bwareaopen(forestMask, 100);
cleanMask = imopen(cleanMask, strel('disk',4));
\end{lstlisting}

\paragraph{Popis:}
Toto je první fáze čištění masky od zbytkové kresby.
\texttt{bwareaopen(..., 100)}: Odstraní všechny malé bílé objekty (šum) menší než 100 pixelů.
\texttt{imopen(..., strel('disk',4))}: Provede morfologickou operaci „otevření“. Ta se skládá z eroze následované dilatací. Tento krok efektivně odstraní tenké čáry (cesty, silnice), které by mohly být součástí masky.

% --- Krok 5.2 ---
\subsubsection*{Krok 5.2: Plnění děr (s výjimkou velkých)}

\paragraph{Kód:}
\noindent
\begin{lstlisting}[language=Matlab]
filledMask = imfill(cleanMask, 'holes');
holes = filledMask & ~cleanMask;
largeHoles = imopen(holes, strel('disk',13));
finalMask = filledMask & ~largeHoles;
\end{lstlisting}

\paragraph{Popis:}
Druhá, složitější fáze čištění, která má za cíl vyplnit díry po cestách a textech uvnitř lesa, ale zároveň zachovat průseky.
\texttt{imfill(..., 'holes')}: Nejprve se vyplní všechny díry v masce.
\texttt{holes = ...}: Vytvoří se pomocná maska, která obsahuje pouze ty díry, které byly v předchozím kroku vyplněny.
\texttt{largeHoles = imopen(...)}: Na masce děr se provede morfologické otevření. To odstraní malé a tenké díry (původní cesty a texty), ale zachová velké a kompaktní díry (průseky).
\texttt{finalMask = filledMask \& \textasciitilde largeHoles}: Finální maska vznikne tak, že se z kompletně vyplněné masky (\texttt{filledMask}) „odečtou“ pouze ty velké díry (\texttt{largeHoles}).

% --- Krok 6 ---
\subsubsection*{Krok 6: Export pixelových souřadnic}

\paragraph{Kód:}
\noindent
\begin{lstlisting}[language=Matlab]
[rows, cols] = find(finalMask);
vysledneSouradnice = [rows, cols];
\end{lstlisting}

\paragraph{Popis:}
Ačkoliv finálním výstupem pro Python skript je rastrový \texttt{.png} soubor, tento krok plní původní požadavek zadání (,,Výsledek odevzdejte jako pole pixelových souřadnic'').
Funkce \texttt{find(finalMask)} najde všechny nenulové (bílé) pixely v \texttt{finalMask} a vrátí jejich řádkové (\texttt{rows}) a sloupcové (\texttt{cols}) indexy.
Tyto dva vektory jsou spojeny do jedné N $\times$~ 2 matice \texttt{vysledneSouradnice}, která představuje požadované pole pixelových souřadnic lesa.

% --- Krok 7 ---
\subsubsection*{Krok 7: Oříznutí a uložení}

\paragraph{Kód:}
\noindent
\begin{lstlisting}[language=Matlab]
finalMask = finalMask(y_min:y_max, x_min:x_max);
imwrite(finalMask, 'maska_std.png');
\end{lstlisting}

\paragraph{Popis:}
Jsou definovány souřadnice ořezového boxu ($x_{min}$, $y_{min}$, $x_{max}$, $y_{max}$).
Finální binární maska (\texttt{finalMask}) je oříznuta na definovanou oblast zájmu.
Výsledná oříznutá maska je uložena jako soubor \filename{maska\_std\_cropped.png}. Tento soubor (obsahující pixely 0 a 1) slouží jako vstup pro navazující Python skript \filename{GeoPackageUTM.py}.

\begin{figure}[p!]
    \centering
    % Můžete si upravit šířku, např. 0.8\textwidth je 80% šířky textu
    \includegraphics[width=1\textwidth]{maska_std_cropped.png} 
    \caption{Výsledná binární maska lesa}
    \label{fig:maska}
\end{figure}

\begin{figure}[p!]
    \centering
    % Můžete si upravit šířku, např. 0.8\textwidth je 80% šířky textu
    \includegraphics[width=1\textwidth]{overlay_cropped.jpg} 
    \caption{Překryv masky nad původní mapou}
    \label{fig:overlay}
\end{figure}


%%%%%%%%%%%%%%%%%%%%%%%%%%%%%%%%%%%%%%%%%%%%%%%%%%%%%%%%%%%%%%%%%%%%%%%%%%%%%%%%
% ZDE ZAČÍNÁ 2. ČÁST
%%%%%%%%%%%%%%%%%%%%%%%%%%%%%%%%%%%%%%%%%%%%%%%%%%%%%%%%%%%%%%%%%%%%%%%%%%%%%%%%

\subsection*{Popis skriptu \filename{GeoPackageUTM.py}}

Tento skript navazuje na \filename{segmentace\_std.m}. Jeho úkolem je vzít surovou binární masku (\filename{maska\_std\_cropped.png}), provést její georeferencování, přeprojektovat ji do systému UTM a uložit výsledek jako finální vektorovou vrstvu ve formátu GeoPackage.
Pro finální transformaci do systému UTM byly porovnány dvě metody: vektorová transformace \texttt{geopandas.to\_crs()} a rastrová reprojekce \texttt{rasterio.reproject()}. Vizuální kontrola nad ortofoto podklady ukázala, že metoda \texttt{geopandas.to\_crs()} zavedla do dat systematický posun o velikosti přibližně 10 metrů. Z tohoto důvodu byla pro finální pipeline zvolena metoda \texttt{rasterio.reproject()}, jejíž výstup přesně lícoval s podkladovými mapami.
\subsubsection*{Část 1: Nastavení a georeference}

Tato část nastavuje všechny potřebné parametry a provádí první georeferenční výpočty.

\paragraph{Krok 1.1: Definice \texttt{control\_points}, souborů a CRS}
\paragraph{Kód:}
\noindent
\begin{lstlisting}[language=Python]
control_points = [...]
\end{lstlisting}
\paragraph{Popis:}
Toto je klíčový seznam 5 identických bodů (GCP). Každý bod obsahuje pár souřadnic:
\begin{itemize}
    \item Pixelové souřadnice (\texttt{pixel\_x}, \texttt{pixel\_y}) odečtené ze souboru \filename{maska\_std\_cropped.png}.
    \item Souřadnice v systému S1942 (\texttt{s1942\_x}, \texttt{s1942\_y}) odečtené z mřížky na původní topografické mapě (např. 3457000, 5595000).
\end{itemize}

\paragraph{Kód:}
\noindent
\begin{lstlisting}[language=Python]
INPUT_RASTER_RAW
OUTPUT_GPKG
TARGET_CRS_S42
TARGET_CRS_UTM
\end{lstlisting}
\paragraph{Popis:}
Jsou nastaveny cesty k vstupnímu \filename{.png} a výstupnímu \filename{.gpkg}. \texttt{TARGET\_CRS\_S42} (EPSG:28403) je CRS, který používáme pro georeferencování. \texttt{TARGET\_CRS\_UTM} (EPSG:32633) je náš finální systém.

\paragraph{Krok 1.2: Výpočet transformace}
\paragraph{Kód:}
\noindent
\begin{lstlisting}[language=Python]
new_transform_S42 = from_gcps(gcps)
\end{lstlisting}
\paragraph{Popis:}
Body \texttt{control\_points} jsou převedeny do formátu \texttt{GroundControlPoint}, který knihovna Rasterio vyžaduje. Funkce \texttt{from\_gcps} z nich vypočítá afinní transformační matici (\texttt{new\_transform\_S42}). Tato matice matematicky popisuje vztah mezi pixely a S-42 souřadnicemi.

\paragraph{Krok 1.3: Načtení dat}
\paragraph{Kód:}
\noindent
\begin{lstlisting}[language=Python]
with rasterio.open(INPUT_RASTER_RAW) as src:
    data = src.read(1)
\end{lstlisting}
\paragraph{Popis:}
Je otevřen vstupní soubor \filename{maska\_std\_cropped.png}. Protože \filename{.png} není georeferencovaný, je potlačeno varování \texttt{NotGeoreferencedWarning}. Pomocí \texttt{src.read(1)} jsou načtena surová pixelová data (pole hodnot 0 a 1) z prvního pásma do proměnné \texttt{data}.

\subsubsection*{Část 2: Reprojekce (S1942 $\to$ UTM)}

Toto je klíčová část skriptu, která provádí transformaci dat pomocí metod Rasterio.

\paragraph{Krok 2.1: Výpočet cílových parametrů}
\paragraph{Kód:}
\noindent
\begin{lstlisting}[language=Python]
dst_transform, dst_width, dst_height = calculate_default_transform(...)
\end{lstlisting}
\paragraph{Popis:}
Abychom mohli data přeprojektovat, musíme definovat „plátno“ v cílovém systému UTM. Tato funkce vypočítá novou transformační matici (\texttt{dst\_transform}) a nové rozměry (\texttt{dst\_width}, \texttt{dst\_height}) pro rastr v UTM tak, aby přesně pokryl původní oblast.

\paragraph{Krok 2.2: Příprava prázdného pole}
\paragraph{Kód:}
\noindent
\begin{lstlisting}[language=Python]
destination_array = np.empty(...)
\end{lstlisting}
\paragraph{Popis:}
V paměti je vytvořeno nové, prázdné numpy pole (\texttt{destination\_array}) s rozměry vypočítanými v předchozím kroku.

\paragraph{Krok 2.3: Reprojekce}
\paragraph{Kód:}
\noindent
\begin{lstlisting}[language=Python]
reproject(...)
\end{lstlisting}
\paragraph{Popis:}
Toto je jádro celého procesu. Funkce \texttt{reproject} vezme zdrojová data (\texttt{data}), zdrojovou transformaci (\texttt{new\_transform\_S42}) a zdrojový CRS (\texttt{TARGET\_CRS\_S42}) a „překreslí“ je do cílového pole (\texttt{destination\_array}) pomocí cílové transformace (\texttt{dst\_transform}) a CRS (\texttt{TARGET\_CRS\_UTM}).
\texttt{Resampling.nearest} je zásadní, protože zajišťuje, že naše hodnoty 0 a 1 zůstanou zachovány a nebudou zprůměrovány (což by se stalo např. při \texttt{Resampling.bilinear}).

\subsubsection*{Část 3: Polygonizace již v UTM}

Nyní, když máme v paměti rastr, který je již správně přeprojektovaný do UTM, můžeme z něj vytvořit vektory.

\paragraph{Krok 3.1: Vytvoření masky}
\paragraph{Kód:}
\noindent
\begin{lstlisting}[language=Python]
mask = (destination_array == 1)
\end{lstlisting}
\paragraph{Popis:}
Vytvoří se jednoduchá booleanská maska, která vybírá pouze pixely lesa (s hodnotou 1).

\paragraph{Krok 3.2: Polygonizace}
\paragraph{Kód:}
\noindent
\begin{lstlisting}[language=Python]
results = rasterio.features.shapes(...)
\end{lstlisting}
\paragraph{Popis:}
Funkce \texttt{shapes} projde reprojektované pole \texttt{destination\_array} a „obkreslí“ všechny souvislé oblasti pixelů s hodnotou 1.
Jelikož jako \texttt{transform} používáme \texttt{dst\_transform} (UTM transformaci), výsledné geometrie (polygony) jsou generovány rovnou v souřadnicích UTM.

\subsubsection*{Část 4: Uložení do GeoPackage}

Finální krok vezme polygony v UTM a uloží je.

\paragraph{Krok 4.1: Kontrola a vytvoření GeoDataFrame}
\paragraph{Kód:}
\noindent
\begin{lstlisting}[language=Python]
gdf_utm = gpd.GeoDataFrame(...)
\end{lstlisting}
\paragraph{Popis:}
Seznam polygonů je načten do GeoDataFrame.\newpage

\paragraph{Krok 4.2: Uložení}
\paragraph{Kód:}
\noindent
\begin{lstlisting}[language=Python]
gdf_utm.to_file(OUTPUT_GPKG, driver='GPKG')
\end{lstlisting}
\paragraph{Popis:}
Finální GeoDataFrame je uložen do souboru \filename{les\_final\_UTM.gpkg}. Tento soubor již obsahuje vektorové polygony lesa ve správném souřadnicovém systému UTM.

\begin{figure}[h!]
    \centering
    % Můžete si upravit šířku, např. 0.8\textwidth je 80% šířky textu
    \includegraphics[width=1\textwidth]{vysledek po importu do qgis.png} 
    \caption{Výsledná vektorová vrstva lesa načtená v QGIS z GeoPackage}
    \label{fig:gpkg}
\end{figure}

\begin{figure}[h!]
    \centering
    % Můžete si upravit šířku, např. 0.8\textwidth je 80% šířky textu
    \includegraphics[width=1\textwidth]{kvalita.png} 
    \caption{Finální ověření správnosti georeferencování. 
             Vektorová vrstva
             je překryta přes podkladový ortofoto snímek. 
             Je patrné, že vypočtené hranice
             lícují se skutečným lesním porostem, 
             což potvrzuje úspěšnou reprojekci do systému UTM.}
    \label{fig:overlay_kontrola}
\end{figure}

\end{document}
%%%%%%%%%%%%%%%%%%%%%%%%%%%%%%%%%%%%%%%%%%%%%%%%%%%%%%%%%%%%%%%%%%%%%%%%%%%%%%%%
% KONEC DOKUMENTU
%%%%%%%%%%%%%%%%%%%%%%%%%%%%%%%%%%%%%%%%%%%%%%%%%%%%%%%%%%%%%%%%%%%%%%%%%%%%%%%%